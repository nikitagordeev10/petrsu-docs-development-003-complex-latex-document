\section{Математический текст}

\subsection{Математические символы}

В математике существует достаточно много различных символов! В формуле \eqref{eq:mathematical_symbols}, к которым можно получить доступ прямо с клавиатуры: 

\begin{equation} \label{eq:mathematical_symbols}
+ - = ! / ( ) [ ] < > | ' : *.
\end{equation}

\subsection{Греческие буквы}
\begin{equation}
\alpha, \beta, \gamma, \pi, \Pi, \phi, \varphi, \mu, \Phi, \varPhi.
\end{equation}

\subsection{Математические операторы}
\begin{equation}\label{eq:test}
\cos (2\theta) = \cos^2 \theta - \sin^2 \theta. 
\end{equation}

\begin{equation}
\lim\limits_{x \to \infty} \exp(-x) = 0. 
\end{equation}

\begin{equation}
a \bmod b. 
\end{equation}

\begin{equation}
x \equiv a \pmod{b}.
\end{equation}

\subsection{Степени и индексы}
Степени и индексы эквивалентны верхним и нижним индексам в обычном текстовом режиме. В формуле \eqref{eq:degrees_and_indices}, к которым можно получить доступ прямо с клавиатуры: 

\begin{equation}\label{eq:degrees_and_indices}
k_{n+1} = n^2 + k_n^2 - k_{n-1} 
\end{equation}

\begin{equation}
n^{22} 
\end{equation}

\begin{equation}
f(n) = n^5 + 4n^2 + 2 |_{n=17} 
\end{equation}

\subsection{Дроби и биномы}
\begin{equation}
\frac{n!}{k!(n-k)!} = \binom{n}{k} 
\end{equation}

\begin{equation}
\frac{\frac{1}{x}+\frac{1}{y}}{y-z} 
\end{equation}

\begin{equation}
x^\frac{1}{2} % no error
\end{equation}

\subsection{Непрерывные дроби}
\begin{equation}
  x = a_0 + \cfrac{1}{a_1 
          + \cfrac{1}{a_2 
          + \cfrac{1}{a_3 + \cfrac{1}{a_4} } } }
\end{equation}

\subsection{Умножение двух чисел}

\begin{equation}
\frac{
    \begin{array}[b]{r}
      \left( x_1 x_2 \right)\\
      \times \left( x'_1 x'_2 \right)
    \end{array}
  }{
    \left( y_1y_2y_3y_4 \right)
  }
\end{equation}

\subsection{Корни}
\begin{equation}
\sqrt{\frac{a}{b}}
\end{equation}

\begin{equation}
\sqrt[n]{1+x+x^2+x^3+\dots+x^n}
\end{equation}

\subsection{Ряды и интегралы}
\begin{equation}
\sum_{i=1}^{10} t_i
\end{equation}

\begin{equation}
\displaystyle\sum_{i=1}^{10} t_i
\end{equation}

\begin{equation}
\int_0^\infty e^{-x}\,\mathrm{d}x
\end{equation}

\begin{equation}
\sum_{\substack{
   0<i<m \\
   0<j<n
  }} 
 P(i,j)
\end{equation}

\begin{equation}
\int\limits_a^b
\end{equation}

\subsection{Скобки, фигурные скобки и разделители}
\begin{equation}
( a ), [ b ], \{ c \}, | d |, \| e \|, \langle f \rangle, \lfloor g \rfloor, \lceil h \rceil, \ulcorner i \urcorner, / j \backslash
\end{equation}

\subsection{Автоматическое определение размеров}
\begin{equation}
\left(\frac{x^2}{y^3}\right)
\end{equation}

\begin{equation}
P\left(A=2\middle|\frac{A^2}{B}>4\right)
\end{equation}

\begin{equation}
\left\{\frac{x^2}{y^3}\right\}
\end{equation}

\begin{equation}
\left.\frac{x^3}{3}\right|_0^1
\end{equation}

\subsection{Ручное определение размеров}
\begin{equation}
( \big( \Big( \bigg( \Bigg(
\end{equation}

\begin{equation}
\frac{\mathrm d}{\mathrm d x} \left( k g(x) \right)
\end{equation}

\begin{equation}
\frac{\mathrm d}{\mathrm d x} \big( k g(x) \big)
\end{equation}

\subsection{Матрицы и массивы}
\begin{equation}
A_{m,n} = 
 \begin{pmatrix}
  a_{1,1} & a_{1,2} & \cdots & a_{1,n} \\
  a_{2,1} & a_{2,2} & \cdots & a_{2,n} \\
  \vdots  & \vdots  & \ddots & \vdots  \\
  a_{m,1} & a_{m,2} & \cdots & a_{m,n} 
 \end{pmatrix}
\end{equation}

\begin{equation}
M = \begin{bmatrix}
       \frac{5}{6} & \frac{1}{6} & 0           \\[0.3em]
       \frac{5}{6} & 0           & \frac{1}{6} \\[0.3em]
       0           & \frac{5}{6} & \frac{1}{6}
     \end{bmatrix}
\end{equation}

\begin{equation}
M = \bordermatrix{~ & x & y \cr
                  A & 1 & 0 \cr
                  B & 0 & 1 \cr}
\end{equation}

\subsection{Фрагмент книги}

The well known Pythagorean theorem \(x^2 + y^2 = z^2\) was 
proved to be invalid for other exponents. 
Meaning the next equation has no integer solutions:

\[ x^n + y^n = z^n \]

Find the difference quotient of $f(x)$ when $f(x)=x^3$.

We proceed as demonstrated in the lab manual; assuming that $h\ne 0$ 
we have
\begin{align*}
    \frac{f(x+h)-f(x)}{h} & =  \frac{(x+h)^3-x^3}{h}   \\
                          & =  \frac{x^3+3x^2h+3xh^2+h^3 - x^3}{h}\\
                          & =  \frac{3x^2h+2xh^2+h^3}{h}\\
                          & =  \frac{h(3x^2+2xh+h^2)}{h}\\
                          & =  3x^2+2xh+h^2
\end{align*} 

Using the definition of the derivative, we have
\begin{align*}
            f'(x)           &= \lim_{h\rightarrow 0}\frac{(x+h)^{1/4}-x^{1/4}}{h}   \\
                            &=  \lim_{h\rightarrow 0}\frac{(x+h)^{1/4}-x^{1/4}}{h}\cdot \frac{((x+h)^{1/4}+x^{1/4})((x+h)^{1/2}+x^{1/2})}{((x+h)^{1/4}+x^{1/4})((x+h)^{1/2}+x^{1/2})}\\
                            &=  \lim_{h\rightarrow 0}\frac{(x+h)-x}{h((x+h)^{1/4}+x^{1/4})((x+h)^{1/2}+x^{1/2})}    \\  
                            &=  \lim_{h\rightarrow 0}\frac{1}{((x+h)^{1/4}+x^{1/4})((x+h)^{1/2}+x^{1/2})}   \\
                            &= \frac{1}{(x^{1/4}+x^{1/4})(x^{1/2}+x^{1/2})} \\
                            &=  \frac{1}{(2x^{1/4})(2x^{1/2})}  \\
                            &=  \frac{1}{4x^{3/4}}  \\
                            &=  \frac{1}{4}x^{-3/4}
\end{align*}
Note: the key observation here is that
\begin{align*}
    a^4-b^4 &= (a^2-b^2)(a^2+b^2)   \\
        &= (a-b)(a+b)(a^2+b^2), 
\end{align*}
with 
\[
    a = (x+h)^{1/4}, \qquad b = x^{1/4},
\]
which allowed us to rationalize the denominator.

In physics, the mass-energy equivalence is stated 
by the equation \begin{equation}E=mc^2\end{equation} discovered in 1905 by Albert Einstein.

\clearpage