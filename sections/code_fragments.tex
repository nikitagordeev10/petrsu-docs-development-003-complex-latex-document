\section{Фрагменты Программного Кода}

Функция, для работы с матрицами представлена на Листинге \ref{lst:PythonExample1}. 

\begin{lstlisting}[
  label={lst:PythonExample1},
  language=Python, 
  caption=Python example]
import numpy as np
    
def incmatrix(genl1,genl2):
    m = len(genl1)
    n = len(genl2)
    M = None #to become the incidence matrix
    VT = np.zeros((n*m,1), int)  #dummy variable
    
    #compute the bitwise xor matrix
    M1 = bitxormatrix(genl1)
    M2 = np.triu(bitxormatrix(genl2),1) 

    for i in range(m-1):
        for j in range(i+1, m):
            [r,c] = np.where(M2 == M1[i,j])
            for k in range(len(r)):
                VT[(i)*n + r[k]] = 1;
                VT[(i)*n + c[k]] = 1;
                VT[(j)*n + r[k]] = 1;
                VT[(j)*n + c[k]] = 1;
                
                if M is None:
                    M = np.copy(VT)
                else:
                    M = np.concatenate((M, VT), 1)
                
                VT = np.zeros((n*m,1), int)
    
    return M
\end{lstlisting}

\clearpage

Функция, для работы с матрицами представлена на Листинге \ref{lst:PythonExample2}.

\lstset{style=mystyle1}
\begin{lstlisting}[
  label={lst:PythonExample2},
        language=Python, 
        caption=Python example 2]
import numpy as np
    
def incmatrix(genl1,genl2):
    m = len(genl1)
    n = len(genl2)
    M = None #to become the incidence matrix
    VT = np.zeros((n*m,1), int)  #dummy variable
    
    #compute the bitwise xor matrix
    M1 = bitxormatrix(genl1)
    M2 = np.triu(bitxormatrix(genl2),1) 

    for i in range(m-1):
        for j in range(i+1, m):
            [r,c] = np.where(M2 == M1[i,j])
            for k in range(len(r)):
                VT[(i)*n + r[k]] = 1;
                VT[(i)*n + c[k]] = 1;
                VT[(j)*n + r[k]] = 1;
                VT[(j)*n + c[k]] = 1;
                
                if M is None:
                    M = np.copy(VT)
                else:
                    M = np.concatenate((M, VT), 1)
                
                VT = np.zeros((n*m,1), int)
    
    return M
\end{lstlisting}
\clearpage


Функция, для работы с матрицами представлена на Листинге \ref{lst:PythonExample3}.

\lstset{style=mystyle2}
\begin{lstlisting}[
    label={lst:PythonExample3},
    language=Python, 
    caption=Python example 3]
import numpy as np
    
def incmatrix(genl1,genl2):
    m = len(genl1)
    n = len(genl2)
    M = None #to become the incidence matrix
    VT = np.zeros((n*m,1), int)  #dummy variable
    
    #compute the bitwise xor matrix
    M1 = bitxormatrix(genl1)
    M2 = np.triu(bitxormatrix(genl2),1) 

    for i in range(m-1):
        for j in range(i+1, m):
            [r,c] = np.where(M2 == M1[i,j])
            for k in range(len(r)):
                VT[(i)*n + r[k]] = 1;
                VT[(i)*n + c[k]] = 1;
                VT[(j)*n + r[k]] = 1;
                VT[(j)*n + c[k]] = 1;
                
                if M is None:
                    M = np.copy(VT)
                else:
                    M = np.concatenate((M, VT), 1)
                
                VT = np.zeros((n*m,1), int)
    
    return M
\end{lstlisting}

\clearpage

% \vspace{10mm}
% \begingroup
% \obeylines
% \input{code_snippets/lorem.tex}%
% \endgroup%
% \clearpage

Функция, для работы с матрицами представлена на Листинге \ref{lst:PythonExample4}.

\vspace{10mm}
\lstset{style=mystyle1}
\lstinputlisting[ label={lst:PythonExample4}, caption=Python example 4, language=Python, firstline=17, lastline=24, firstnumber=17, numbers=left]{code_snippets/sort.py}

\clearpage
